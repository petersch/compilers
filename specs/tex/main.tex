\documentclass{article}


\usepackage{czech}
\usepackage[slovak]{babel}
%\usepackage[latin1]{inputenc}
\usepackage[utf8]{inputenc}
\usepackage{graphicx}

%\usepackage{color}
\usepackage[usenames,dvipsnames]{color}
\usepackage{listings}


\begin{document}

\definecolor{lg}{gray}{0.7}

\lstdefinestyle{customc}{
  belowcaptionskip=1\baselineskip,
  breaklines=true,
  frame=L,
  xleftmargin=\parindent,
  language=C,
  showstringspaces=false,
  basicstyle=\footnotesize\ttfamily,
  keywordstyle=\bfseries\color{blue},
  keywords=[2]{read,write,unless,bool,in},
  keywordstyle={[2]\bfseries\color{blue}},
  commentstyle=\color{lg},
  identifierstyle=\color{black},
  stringstyle=\color{red},
}

\lstset{escapechar=@,style=customc}


\centerline{\sc \large Kompilátory - Assignment 1}
\vspace{.5pc}
\centerline{\sc Peter Schmidt}
\vspace{2pc}

\section{Špecifikácia jazyka}
 
    \subsection{Syntax}
        Syntax je podobná jazyku C.
        Každý riadok obsahuje najviac jeden príkaz (priradenie, volanie funkcie alebo niektorú z riadiacich štruktúr),
        a príkazy sa neoddeľujú bodkočiarkami.
        Identifikátory (názvy funkcií a mená premenných) musia spĺňať regulárny výraz \texttt{[\_a-zA-Z][\_a-zA-Z0-9]*}

    \subsection{Premenné}
        Jazyk podporuje 32 bitový celočíselný typ int, polia typu int a booleovský typ bool.
        Jazyk podporuje stringové literály iba pre výstup.
        Premenné majú statický typ, deklarované sú ako v jazyku C.
        Neinicializované premenné majú hodnotu 0 resp. \texttt{false}.
        \begin{lstlisting}
int i, j
int n = 5
        \end{lstlisting}
        Premenné sú platné po koniec bloku v ktorom boli definované.
        Nie je povolené deklarovať premennú s názvom rovnakým ako názov funkcie alebo meno inej platnej premennej.

    \subsection{Operátory}
        Jazyk podporuje operátory pre bežné aritmetické a logické operácie
        \begin{lstlisting}
+ - * / % & | ^
        \end{lstlisting}
        a relácie porovnania
        \begin{lstlisting}
< > <= >= == !=
        \end{lstlisting}

    \subsection{Polia}
        Jazyk podporuje iba polia typu \texttt{int}.
        Polia sú indexované od 0, veľkosť poľa je možné zistiť vstavanou funkciu \texttt{size()}.
        Veľkosť poľa je definovaná v runtime a pole je zrušené pri návrate z funkcie.
        \begin{lstlisting}
int pole[5]
int pole[5][2]
n = pole[i]
        \end{lstlisting}

    \subsection{Funkcie}
        Funkcie sú deklarované a definované rovnako ako v jazyku C.
        \begin{lstlisting}
int fn(int i, int pole[]) {
    return pole[i]
}
        \end{lstlisting}
        Tiež je možné používať externé funkcie, tie je potrebné deklarovať na začiatku zdrojového kódu.
        \begin{lstlisting}
extern int fn(int, int)
        \end{lstlisting}

    \subsection{Vstup a výstup}
        Pre štandardný vstup a výstup slúžia kľúčové slová read a write.
        Načítať je možné iba do premennej typu int, pre výpis je možné navyše používať stringové literály.
        \begin{lstlisting}
read n                  // ak n je typu int, nacita cele cislo
write "vysledok: ", n   // vypise string a cislo n
        \end{lstlisting}

    \subsection{Riadiace štruktúry}
        Jazyk podporuje for a while pre cykly a if a unless pre podmienené vetvenie.
        Iteračná premenná for-cyklu je vždy typu int, čo nie je potrebné deklarovať.
        Pre potreby cyklu sa používa range (ako v Ruby), inkluzívny 1..n, resp. exkluzívny 0...size(pole)

        \begin{lstlisting}
for i in 0...size(pole) {
    // do something with pole[i]
}

if n < m {
    // ...
}
else {
    // ...
}
        \end{lstlisting}
        
        
        
\section{Príklady}

    \subsection{Triedenie}
        \begin{lstlisting}
void merge(int[] in1, int[] in2, int[] out) {
    int p1 = 0, p2 = 0
    for i in 0...size(out) {
         if in1[p1] <= in2[p2] {
             out[i] = in1[p1]
             ++p1
         }
         else {
             out[i] = in2[p2]
             ++p2
         }
    }
}

void merge_sort(int[] pole) {
    if size(pole) == 1 {
        return
    }
    
    int odd = (size(pole) % 2 == 0) ? 0 : 1
    int sub1[size(pole) / 2 + odd]
    int sub2[size(pole) / 2]
    
    for i in 0...size(sub1) {
        sub1[i] = pole[i]
    }    
    for i in 0...size(sub2) {
        sub2[i] = pole[i + size(sub1)]
    }
    
    merge_sort(sub1)
    merge_sort(sub2)
    merge(pole, sub1, sub2)
}

void main() {
    int n
    read n
    
    int pole[n]
    for i in 0...n {
        read pole[i]
    }
    
    merge_sort(pole)
    
    for i in 0...n {
        write pole[i]
    }
}

        \end{lstlisting}

    \subsection{Prvočísla}
        \begin{lstlisting}
void main() {
    int n
    read n

    int pole[n-2]
    for i in 0...size(pole) {
        pole[i] = i+2
    }
    
    for i in 0...size(pole) {
        unless pole[i] == 0 {
            for j in (i+1)...size(pole) {
                if pole[j] % i == 0 {
                    pole[j] = 0
                }
            }
        }
    }
    
    for i in 0...size(pole) {
        unless pole[i] == 0 {
            write pole[i]
        }
    }
}

        \end{lstlisting}

    \subsection{Súvislosť grafu}
        \begin{lstlisting}
int findset(int[] pole, int i) {
    int iset = i
    while pole[iset] != iset {
        iset = pole[iset]
    }
    while pole[i] != iset {
        pole[i] = iset
        i = pole[i]
    }
    return iset
}

int union(int[] pole, int i, int j) {
    int iset = findset(pole, i)
    pole[iset] = findset(pole, j)
    return pole[iset]
}

void main() {
    int n, m
    read n, m

    int sets[n]
    for i in 0...n {
        sets[i] = i
    }

    for i in 0...m {
        int u, v
        read u, v
        union(sets, u, v)
    }

    bool connected = true
    for i in 1...n {
        unless sets[i-1] == sets[i] {
            connected = false
        }
    }

    if connected {
        write "ANO\n"
    }
    else {
        write "NIE\n"
    }
}

        \end{lstlisting}



\end{document}
